\newcommand{\package}{\emph}

\setcounter{chapter}{1}
\setcounter{section}{0}
\section{Allometric scaling}
We know (from lecture and an article ‘Some Dynamic Aspects of HSC’) that a number of active HSC follows negative allometry (power law) to mass of mammal with scaling exponent $0.75$. 

\[ N_{sc} \approx M^{0.75} \]

That means that to get to know the number of Active HSC in any mammal we just need to calibrate that power law by experimentally getting to know $N_{sc}$ and $M$ of any mammal. It was done for cats and humans, so for humans $M=70$ and $N_{SC}=385$, so

\[ N_{sc}= N_{sc_0}\times M^{0.75} \]
\[ 385=N_{sc_0}\times 70^{0.75} \]
\[N_{sc_0}=15.9088\]

Now, we can estimate the number of active HSC in hamster (average M=100gm = 0.1 kg), so

\[ N_{sc}= 15.9088 \times 0.1^{0.75} \approx 2.3 \]

Same for blue whale (M =100 000 kg, \url{http://www.marinemammalcenter.org}), so

\[ N_{sc}= 15.9088 \times 100000^{0.75} \approx 89 461 \]

The range of total number of HSC is the same for all mammals and approximately equal to 11000...22000, so it is the same for hamster and blue whale.

\setcounter{chapter}{2}
\setcounter{section}{0}
\section{Hematopoietic multicompartment model}
$K$ – number of compartments (successes, i.e. cell manages to move to k-th compartment) 

$E$ – probability of differentiation.

$D$ – number of additional non-differentiating divisions which cell underwent before entering k-th compartment (failures, i.e cell failed to move to the next compartment).

So, $K+D$  is equal to the total number of cell divisions before k-compartment.
$D$ follows negative binomial distribution (in our case $P(D)=NBin$ is a probability of $D$ failures (staying in the same compartment) before $k$ successes (jumping into next compartment) occurred).

\subsection{a}

So, we know that expectation of $NBin$ distribution (i.e expected number of staying in the same compartment before reaching k-th compartment) can be calculated as follows:

\[ E[D] = \sum\limits_{d=0}^{\inf} dP(D=d) = \sum\limits_{d=0}^{\inf} d \left( \begin{matrix}
  k+d-1 \\
  d
 \end{matrix} \right) (1-\epsilon)^{d}\epsilon^k  \]
 
 \[ \sum\limits_{d=1}^{\inf} d \frac{(k+d-1)!}{d! (k-1)!}(1-\epsilon)^d\epsilon^k = \sum\limits_{d=1}^{\inf}\frac{(k+d-1)}{(d-1)!(k-1)!}(1-\epsilon)^d\epsilon^k\]

\[ k\frac{1-\epsilon}{\epsilon} \sum\limits_{d=1}^{\inf} \frac{(k+d-1)!}{(d-1)!(k)!}(1-\epsilon)^{d-1}\epsilon^{k+1}  \]

by change of variable $z=d-1$, we have:

\[ E[D] = k\frac{1-\epsilon}{\epsilon}\sum\limits_{z=0}^{\inf} \frac{k+z}!{z!k!}(1-\epsilon)^z\epsilon^{k+1} = k\frac{1-\epsilon}{\epsilon} \sum\limits_{z=0}^{\inf} \left( \begin{matrix}
  (k+1)+z-1 \\
  z
 \end{matrix} \right) (1-\epsilon)^z\epsilon^{k+1} = k\frac{1-\epsilon}{\epsilon}\]
 
 So, for reaching $k=30$ compartment (the last one, i.e to differentiate to a red blood cell) with e=0.85, we will expect the following number of identical divisions:

\[ E[D]=k\frac{1-\epsilon}{\epsilon}=\frac{(30\times(1-0.85))}{0.85}=5.29 \]

\subsection{b}

During the leukemia the self renewal probability $(1-e)$ increases by 15\%, (so we can calculate a new $e=0.827$) and in that case we will expect the following number of non-differentiated divisions till red blood cell formation:

\[ E[D]=k\frac{1-\epsilon}{\epsilon}=\frac{(30\times(1-0.827))}{0.827}=6.25 \]

To compute the number of leukemic cells, one should solve the following equation:

\[ \frac{N_k}{N_{k-1}} = \frac{2\epsilon}{2\epsilon - 1} \times \frac{r_{k-1}}{r_k}\]

In the stationary condition, we assume $\frac{r_k}{r_{k-1}} = r$ and we can assume that $r$ remains constant at $r=1.27$ as claimed in (Dingli er al, \textit{"Compartmental Architecture and Dynamics of Hematopoiesis"}, PlosOne, 2007, e345).
Then:

\[ \frac{N_k}{N_{k-1}} = \left( \frac{2\epsilon}{2\epsilon-1} \times \frac{1}{r} \right)^k N_0 = N_k \]

\[ \left( \frac{1.654}{0.654}*\frac{1}{1.27} \right)^{30} * 1 =  N_{30} = 917,464,597  \]


\setcounter{chapter}{3}
\setcounter{section}{0}
\section{Treatment of chronic myeloid leukemia}

\begin{align*}
x'_0 &= [\lambda x_0 - d_0]x_0 \rightarrow \text{quiescent stem cells}\\
x'_1 &= a_1x_0 - d_1x_1 \rightarrow \text{actively proliferating stem cells}\\
x'_2 &= a_1x_1 - d_2x_2 \rightarrow \text{early differentiated cells}\\
x'_3 &= a_3x_2 - d_3x_3 \rightarrow \text{fully differentiated blood cells}
\end{align*}
\[ 0 < d_1 \ll d_2 \ll d_3 \text{~~~~~~~~~~} \frac{a_i}{d_i} > 1  \text{~~~~~~~~~~} \forall i \]

at steady state, all $(\vec{x'_i}) = \vec{0}$. In the trivial case: $x^*_0 = 0 \leftrightarrow x^*_1 = 0 \leftrightarrow x^*_2 = 0 \leftrightarrow x^*_3 = 0$
Other case from:

\begin{align*}
x^*_0 &= \frac{d_0}{\lambda}\\
x^*_1 &= \frac{a_1}{d_1}\frac{d_0}{\lambda}\\
x^*_2 &= \frac{a_2}{d_2}\frac{a_1}{d_1}\frac{d_0}{\lambda}\\
x^*_3 &= \frac{a_3}{d_3}\frac{a_2}{d_2}\frac{a_1}{d_1}\frac{d_0}{\lambda}\\
\end{align*}

Thus, the most abundandt cell type at steady state is $x_3$ since $\frac{a_i}{d_i} > 1 \text{~~~} \forall i $

Proof exemplarily for: \[x^*_3 > x^*_2 : \frac{a_3}{d_3}  \underbrace{\frac{a_2}{d_2}\frac{a_1}{d_1}\frac{d_0}{\lambda}}_\text{cancels out to 1} >  \underbrace{\frac{a_2}{d_2}\frac{a_1}{d_1}\frac{d_0}{\lambda}}_\text{cancels out to 1}\]

So, \[\frac{a_3}{d_3}>1\]

\subsection{b}

From the non trivial steady state above, if $x_1 = x^*_1$, then all other $x_i$ are also in the
steady state $x^*_i$ at $t = 0$. Since imatinib treatment does not change $x'_0$, and $x_0$ is in the steady state at  $t = 0 : x'_0 = 0 \leftrightarrows x_0 = \frac{d_0}{\lambda}$ , it can be seen as constant. That leads to:

\[ x'_1 = a'_1x_0 - d_1x_1 = a'_1\frac{d_0}{\lambda} -d_1x_1  \]

The solution for this is: 

\begin{align*}
\frac{dx_1}{dt} &= -d_1x_1 \text{general solution}\\
\frac{dx_1}{x_1} &= -d_1dt \\
\log x_1 &= d_1t+C \\
x_1(t) &= e^{-d_1t}\times C
\end{align*}

In steady state conditions, when $x_1(0) = \frac{a_1d_0}{d_1\lambda}$, it is:

\[ x_1(0) = e^{-d_1(0)} \cdot C = \frac{a_1 \cdot d_0}{ d_1 \cdot \lambda}\]
\[ C = \frac{a_1d_0}{d_1\lambda} \]

To retrieve solutions in nonhomogeneous differential equations, we have to take in account for partial solutions:

\[ y'(x) + a(x)y(x) = s(x) \]


\[
 y(p) = \begin{dcases}
		\frac{s}{a}  & a \neq 0,  a \wedge s = constant\\
		\int s(x) e^{\int a(x)dx} & a(x) \wedge s(x) \neq constant
        \end{dcases}
\]

Hence,

\[ \rightarrow \frac{s}{a} = \frac{a'_1d_0}{\lambda d_1} \]
\[ \rightarrow x_1(t) = e^{d_1t}\frac{a_1d_0}{\lambda d_1} + \frac{a'_1d_0}{\lambda d_1} \]

This scales with $\frac{1}{d_1}$ (average life time of $x_1$-cells), since almost no $x_1$ cells differentiate from $x_0$ ($a'_1 \ll a_1$) and the death rate $d_1$ is constant:

\[ x_1(t) = \frac{1}{d_1} \left(e^{d_1t}\frac{a_1d_0}{\lambda}\frac{a'_1d_0}{\lambda}\right)\]

\subsection{c}

The bi-phasic decline of the number of leukemic blood cells $x3$ can be explained in the
following way: the first slope describes the exponentially decline of the early differentiated leukemic
cells until they reach the new steady state $x^*_2 = \frac{a'_2}{d_2} \cdot x^*_1$ with the leukemic progenitors. Now, the decline of leukemic blood cells follows the decline of leukemic progenitors. 

Thus, the death rates $d1$ and $d2$ can be directly determined from the graph:
\begin{itemize}
\item $d1 = 0.004$  death rate of leukemic progenitors during imatinib therapy
\item $d2 = 0.03$  death rate of leukemic early differentiated cells during imatinib therapy
\end{itemize}

\subsection{d}

If the treatment with imatinib is stopped or the cells become resistant to imatinib after a certain time, the leukemic cells will exponentially appear again. Since imatinib does not affect the leukemic stem cells, $x_0$ but only the (in leukemia uncontrolled) differentiating blood cells $x_1$ and $x_2$ (and thus the number of leukemic blood cells $x_3$), a new population of leukemic cells will arise from the stem cell pool with exponential growth until the number of leukemic cells reaches the level of before treatment (or even above this level if the number of leukemic stem cells is not constant but increasing due to a higher proliferation rate).
Same is true if the number of quiescent stem cells has not yet reached steady state at the onset of treatment: the leukemic blood cells will exceed the level from before by far, because, unchanged by the imatinib treatment, the quiescent stem cells will reach steady state and give exponentially rise to leukemic blood cells after the interruption of the treatment.

\setcounter{chapter}{4}
\setcounter{section}{0}
\section{One-dimensional Fokker-Planck equation}

\setcounter{chapter}{5}
\setcounter{section}{0}
\section{Diffusion approximation of the Moran process}

\setcounter{chapter}{6}
\setcounter{section}{0}
\section{Absorption time in the diffusion approximation}
