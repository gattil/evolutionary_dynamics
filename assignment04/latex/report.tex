\newcommand{\package}{\emph}

\setcounter{chapter}{1}
\setcounter{section}{0}
\section{Allometric scaling}
We know (from lecture and an article ‘Some Dynamic Aspects of HSC’) that a number of active HSC follows negative allometry (power law) to mass of mammal with scaling exponent $0.75$. 

\[ N_{sc} \approx M^{0.75} \]

That means that to get to know the number of Active HSC in any mammal we just need to calibrate that power law by experimentally getting to know $N_{sc}$ and $M$ of any mammal. It was done for cats and humans, so for humans $M=70$ and $N_{SC}=385$, so

\[ N_{sc}= N_{sc_0}\times M^{0.75} \]
\[ 385=N_{sc_0}\times 70^{0.75} \]
\[N_{sc_0}=15.9088\]

Now, we can estimate the number of active HSC in hamster (average M=100gm = 0.1 kg), so

\[ N_{sc}= 15.9088 \times 0.1^{0.75} \approx 2.3 \]

Same for blue whale (M =100 000 kg, \url{http://www.marinemammalcenter.org}), so

\[ N_{sc}= 15.9088 \times 100000^{0.75} \approx 89 461 \]

The range of total number of HSC is the same for all mammals and approximately equal to 11000...22000, so it is the same for hamster and blue whale.

\setcounter{chapter}{2}
\setcounter{section}{0}
\section{Hematopoietic multicompartment model}

\setcounter{chapter}{3}
\setcounter{section}{0}
\section{Treatment of chronic myeloid leukemia}

\setcounter{chapter}{4}
\setcounter{section}{0}
\section{One-dimensional Fokker-Planck equation}

\setcounter{chapter}{5}
\setcounter{section}{0}
\section{Diffusion approximation of the Moran process}

\setcounter{chapter}{6}
\setcounter{section}{0}
\section{Absorption time in the diffusion approximation}
