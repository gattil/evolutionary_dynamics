\newcommand{\package}{\emph}

\setcounter{chapter}{1}
\setcounter{section}{0}
\section{Allometric scaling}
We know (from lecture and an article ‘Some Dynamic Aspects of HSC’) that a number of active HSC follows negative allometry (power law) to mass of mammal with scaling exponent $0.75$. 

\[ N_{sc} \approx M^{0.75} \]

That means that to get to know the number of Active HSC in any mammal we just need to calibrate that power law by experimentally getting to know $N_{sc}$ and $M$ of any mammal. It was done for cats and humans, so for humans $M=70$ and $N_{SC}=385$, so

\[ N_{sc}= N_{sc_0}\times M^{0.75} \]
\[ 385=N_{sc_0}\times 70^{0.75} \]
\[N_{sc_0}=15.9088\]

Now, we can estimate the number of active HSC in hamster (average M=100gm = 0.1 kg), so

\[ N_{sc}= 15.9088 \times 0.1^{0.75} \approx 2.3 \]

Same for blue whale (M =100 000 kg, \url{http://www.marinemammalcenter.org}), so

\[ N_{sc}= 15.9088 \times 100000^{0.75} \approx 89 461 \]

The range of total number of HSC is the same for all mammals and approximately equal to 11000...22000, so it is the same for hamster and blue whale.

\setcounter{chapter}{2}
\setcounter{section}{0}
\section{Hematopoietic multicompartment model}
$K$ – number of compartments (successes, i.e. cell manages to move to k-th compartment) 

$E$ – probability of differentiation.

$D$ – number of additional non-differentiating divisions which cell underwent before entering k-th compartment (failures, i.e cell failed to move to the next compartment).

So, $K+D$  is equal to the total number of cell divisions before k-compartment.
$D$ follows negative binomial distribution (in our case $P(D)=NBin$ is a probability of $D$ failures (staying in the same compartment) before $k$ successes (jumping into next compartment) occurred).

\subsection{a}

So, we know that expectation of $NBin$ distribution (i.e expected number of staying in the same compartment before reaching k-th compartment) can be calculated as follows:

\[ E[D] = \sum\limits_{d=0}^{\inf} dP(D=d) = \sum\limits_{d=0}^{\inf} d \left( \begin{matrix}
  k+d-1 \\
  d
 \end{matrix} \right) (1-\epsilon)^{d}\epsilon^k  \]
 
 \[ \sum\limits_{d=1}^{\inf} d \frac{(k+d-1)!}{d! (k-1)!}(1-\epsilon)^d\epsilon^k = \sum\limits_{d=1}^{\inf}\frac{(k+d-1)}{(d-1)!(k-1)!}(1-\epsilon)^d\epsilon^k\]

\[ k\frac{1-\epsilon}{\epsilon} \sum\limits_{d=1}^{\inf} \frac{(k+d-1)!}{(d-1)!(k)!}(1-\epsilon)^{d-1}\epsilon^{k+1}  \]

by change of variable $z=d-1$, we have:

\[ E[D] = k\frac{1-\epsilon}{\epsilon}\sum\limits_{z=0}^{\inf} \frac{k+z}!{z!k!}(1-\epsilon)^z\epsilon^{k+1} = k\frac{1-\epsilon}{\epsilon} \sum\limits_{z=0}^{\inf} \left( \begin{matrix}
  (k+1)+z-1 \\
  z
 \end{matrix} \right) (1-\epsilon)^z\epsilon^{k+1} = k\frac{1-\epsilon}{\epsilon}\]
 
 So, for reaching $k=30$ compartment (the last one, i.e to differentiate to a red blood cell) with e=0.85, we will expect the following number of identical divisions:

\[ E[D]=k\frac{1-\epsilon}{\epsilon}=\frac{(30\times(1-0.85))}{0.85}=5.29 \]

\subsection{b}

During the leukemia the self renewal probability $(1-e)$ increases by 15\%, (so we can calculate a new $e=0.827$) and in that case we will expect the following number of non-differentiated divisions till red blood cell formation:

\[ E[D]=k\frac{1-\epsilon}{\epsilon}=\frac{(30\times(1-0.827))}{0.827}=6.25 \]

To compute the number of leukemic cells, one should solve the following equation:

\[ \frac{N_k}{N_{k-1}} = \frac{2\epsilon}{2\epsilon - 1} \times \frac{r_{k-1}}{r_k}\]

In the stationary condition, we assume $\frac{r_k}{r_{k-1}} = r$ and we can assume that $r$ remains constant at $r=1.27$ as claimed in (Dingli er al, \textit{"Compartmental Architecture and Dynamics of Hematopoiesis"}, PlosOne, 2007, e345).
Then:

\[ \frac{N_k}{N_{k-1}} = \left( \frac{2\epsilon}{2\epsilon-1} \times \frac{1}{r} \right)^k N_0 = N_k \]

\[ \left( \frac{1.654}{0.654}*\frac{1}{1.27} \right)^{30} * 1 =  N_{30} = 917,464,597  \]


\setcounter{chapter}{3}
\setcounter{section}{0}
\section{Treatment of chronic myeloid leukemia}

\setcounter{chapter}{4}
\setcounter{section}{0}
\section{One-dimensional Fokker-Planck equation}

\setcounter{chapter}{5}
\setcounter{section}{0}
\section{Diffusion approximation of the Moran process}

\setcounter{chapter}{6}
\setcounter{section}{0}
\section{Absorption time in the diffusion approximation}
