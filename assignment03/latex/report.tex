\newcommand{\package}{\emph}

\setcounter{chapter}{1}
\setcounter{section}{0}
\section{Chromosomal instability}
\subsection{a}
Calculate three ratios $C= \frac{CIN}{no-CIN}$ and show that $C$ is independent of time
1.	Neutral CIN : if we assume that genes with CIN are neutral (have no fitness advantages or disadvantages) we can conclude that mutation rate from state $A(+-) \rightarrow A(--)$ is $Nu_2$ and from $A (+- CIN) \rightarrow A(-- CIN)$ is $Nu_3$. So we can create a linear ODE system. 
From lecture we know that its solutions for  $X_2$ (i.e. A(--)) and $Y_2$ (i.e. A(-- CIN)) are the following:
\[X_2(t) \approx Nu_1u_2 \times \frac{t^2}{2} \]
\[Y_2(t) \approx u_1u_ct^2\]
So, our rate is


\setcounter{chapter}{2}
\setcounter{section}{0}
\section{Linear process of colonic crypt transformation}

\setcounter{chapter}{3}
\setcounter{section}{0}
\section{Multistage theory}

\setcounter{chapter}{4}
\setcounter{section}{0}
\section{Pathways of carcinogenesis}
\subsection{a}
The probability of the path $ P = 2 \rightarrow 3 \rightarrow 1 $ for three independent mutations occurring after exponentially distributed waiting time $T_i \sim exp(\lambda_i), i = 1,2,3$ is:

\[ P = J_1 \rightarrow \dots \rightarrow J_k = J_2 \rightarrow J_3 \rightarrow J_1  \]
\begin{center}
\begin{tikzpicture}[
back line/.style={densely dotted},
hligh line/.style={preaction={draw=yellow,-,double=yellow,double distance=6\pgflinewidth}},
cross line/.style={preaction={draw=white, -,line width=6pt}}]
\matrix (m) [matrix of math nodes,
row sep=2em, column sep=2em,
text height=1.5ex,
text depth=0.25ex]{
& \{2,3\} & & \{1,2,3\}\\
\{3\}& & \{3,1\}\\
& \{2\} & & \{1,2\}\\
\{0\} & & \{1\} \\
};
\path[->]
(m-1-2) edge [hligh line] (m-1-4) 
(m-2-1) edge (m-1-2) 
(m-2-1) edge [cross line] (m-2-3) 
(m-2-3) edge (m-1-4) 
(m-3-2) edge [hligh line] (m-1-2) edge [back line] (m-3-4)
(m-3-4) edge (m-1-4)
(m-4-1) edge (m-4-3) edge (m-2-1) edge [hligh line] (m-3-2)
(m-4-3) edge [cross line] (m-2-3) edge (m-3-4);
\end{tikzpicture}
\end{center}

\[ \text{Prob(P)} = \prod\limits_{i=1}^{3} \frac{\lambda_{Ji}}{\sum\limits_{J \in \text{Exit}_i} \lambda_J } = \frac{\lambda_2}{\sum\limits_{J \in \text{Exit}_i = 1,2,3} \lambda_J} \times \frac{\lambda_3}{\sum\limits_{J \in \text{Exit}_i = 1,3} \lambda_J}  \times \frac{\lambda_1}{\sum\limits_{J \in \text{Exit}_i = 1} \lambda_J} = \frac{\lambda_2\lambda_3}{(\lambda_1+\lambda_2+\lambda_3)\times(\lambda_3+\lambda_1)}\]

\subsection{b}

All possible genotypes starting from the wt (no mutation occurred) are 8: $\{0\};\{1,2,3\};\{12,23,31\};\{123\}$. Considering 2 out of 3 mutations one will obtain 6 possible pathways.
Then, the expected waiting time is (where $k$ is the number of mutations expected and $p$ the number of pathways):

\[E[T_k] = \sum\limits_{p=1}^{6} \sum\limits_{n=1}^{k=2} \frac{1}{\sum\limits_{J \in \text{Exit}_i} \lambda_J} \times \text{Prob}(P) = \sum\limits_{p=1}^{6} \sum\limits_{n=1}^{k=2} \frac{1}{\sum\limits_{J \in \text{Exit}_i} \lambda_J} \times \prod\limits_{i=1}^{3} \frac{\lambda_{Ji}}{\sum\limits_{J \in \text{Exit}_i} \lambda_J } \]

\setcounter{chapter}{5}
\setcounter{section}{0}
\section{Neutral Wright-Fisher process}

\setcounter{chapter}{6}
\setcounter{section}{0}
\section{Wave approximation}
